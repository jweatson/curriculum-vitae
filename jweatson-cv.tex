%%%%%%%%%%%%%%%%%%%%%%%%%%%%%%%%%%%%%%%%%
% Medium Length Professional CV
% LaTeX Template
% Version 2.0 (8/5/13)
%
% This template has been downloaded from:
% http://www.LaTeXTemplates.com
%
% Original author:
% Trey Hunner (http://www.treyhunner.com/)
% Some modification by Joseph Eatson
%
% Important note:
% This template requires the resume.cls file to be in the same directory as the
% .tex file. The resume.cls file provides the resume style used for structuring the
% document.
%
%%%%%%%%%%%%%%%%%%%%%%%%%%%%%%%%%%%%%%%%%

\documentclass{resume} % Use the custom resume.cls style
\usepackage[a4paper,left=2.5cm,top=2cm,right=2.5cm,bottom=2cm]{geometry} % Document margins
\usepackage{hyperref}
\hypersetup{
    colorlinks=true,
    linkcolor=blue,
    filecolor=magenta,      
    urlcolor=cyan,
    pdftitle={Joseph Eatson CV},
    pdfpagemode=FullScreen,
    }

% Personal details
\name{Joseph Eatson} % Your name
\address{Hicks Building \\ Hounsfield Road \\ Sheffield \\ S3 7RH \\ United Kingdom}
\address{\texttt{\href{mailto:j.w.eatson@sheffield.ac.uk}{j.w.eatson@sheffield.ac.uk}} \\  \texttt{\href{https://orcid.org/0000-0002-5160-8871}{ORCID:0000-0002-5160-8871}}}
\address{University of Sheffield \\ School of Mathematical \& Physical Sciences}

% Hicks Building, Hounsfield Rd, Broomhall, Sheffield S3 7RH


% Begin document
\begin{document}

\begin{rSection}{Research Experience}
	
	\begin{rPoints}{Postdoctoral Researcher, University of Sheffield (2022--2025)}
		\item Investigating the influence of Short-Lived Radioisotopes (SLRs) on star-forming regions, protoplanetary disks and planetesimals through numerical simulations.
		\item Planetesimals are simulated to determine the effect of varying levels of SLR enrichment on final planet volatile content, star-forming regions are simulated to determine the likelihood of enrichment levels.
		\item Star-forming regions and disks are simulated via $N$-body simulations, planetesimals are simulated via a grid-based fluid dynamics and geodynamics model.
		\item Research into water retention after planetesimal heating, and how this affects water content of exoplanets, as well as formation of the solar system.
		\item Current and future work focussed on developing a new open-source multi-fluid geodynamical model as part of an international collaboration with the Forming Worlds Lab, which aims to implement out-gassing, fluid flow, pebble accretion and percolation for a comprehensive planetesimal model. 
		\item Further research into a less conventional SLR enrichment route, AGB ``interlopers'' is also being performed.
		\item Initial results suggest that the solar system is relatively highly enriched, however, conventional mechanisms for enrichment have difficulty explaining $^{60}$Fe enrichment without being extremely proximal to supernovae, hence importance of AGB ``interlopers''. 
	\end{rPoints}
	\begin{rPoints}{PhD Research, University of Leeds (2017--2022)}
		\item Investigated the formation of dust within Colliding Wind Binary (CWB) systems.
		\item Simulating these systems had significant computational and physical challenges, such as the need for extremely large 3-D grids, cooling through dust and plasma emission, and the complexity of simulating highly unstable winds and shocks.
		\item Significant modification of the fluid dynamics code Athena++ performed to achieve this, which was run on the Leeds ARC 4 HPC system.
		\item Simulations performed using a passive scalar dust model included in very large-scale 3D hydrodynamical simulations of CWBs to hundreds of AU.
		\item Results suggested that dust production within the post-shock region of the wind collision region could result in significant quantities of dust being produced, with favourable density and temperature conditions after cooling.
	\end{rPoints}
\end{rSection}

\begin{rSection}{Relevant research skills}
	\begin{rPoints}{}
		\item Research background in star formation, planet formation, early-type stellar winds, planetesimals and star forming regions.
		\item Comprehensive knowledge of multiple programming languages, in particular 8 years of experience in C, C++, Python, and R -- and 4 years of experience using Rust, Fortran and Julia.
		\item 8 years programming experience with high-performance compute systems, experience in writing efficient, highly parallel code, experience in utilising HPC clusters and GPU acceleration.
		\item Familiarity in using and writing extensions for scientific computing projects such as {AMUSE} and {Athena++}.
		\item Experience running and analysing the results of fluid dynamics models.
		\item Experience with writing well-documented code and using version control software such as {Git} for open-source projects.
		\item Highly self-motivated researcher, have written multiple papers as first author, handling the majority of the research, writing and feedback from peer review. 
		\item 4 years experience teaching hands-on lab experiments and computing workshops to undergraduate students, as well as assessing and grading practical work.
	\end{rPoints}
\end{rSection}

\begin{rSection}{Education}
	\begin{rPoints}{University of Leeds (2017--2022)}
		\item Astrophysics PhD -- {\sl Numerical Simulations of Dusty Colliding Wind Binaries}
		\item Date of award -- {21\textsuperscript{st} of November 2022} 
	\end{rPoints}
	\begin{rPoints}{University of Leeds (2013--2017)}
		\item {Master of Physics} -- {with merit}
		\item {Bachelor of Science} -- {2.1}
		\item Date of award -- {30\textsuperscript{th} of June 2017}
	\end{rPoints}
	\begin{rPoints}{Enfield Grammar School (2006-2013)}
		\item A-levels in Mathematics (A), Physics (B) \& History (A)
		\item 13 GCSEs
	\end{rPoints}
\end{rSection}

\begin{rSection}{References}
	{\textbf{Dr. Richard Parker} \hfill \textsl{School of Mathematical \& Physical Sciences, University of Sheffield}} \\
	{\hspace*{0pt}\hfill \texttt{+44 114 222 3560}, \texttt{\href{mailto:r.parker@sheffield.ac.uk}{r.parker@sheffield.ac.uk}}} \\
	% {\textbf{Dr. Tim Lichtenberg} \hfill \textsl{Faculty of Science and Engineering, University of Groningen}} \\ 
	% {\hspace*{0pt}\hfill {+31 50 36 34073}, {\href{mailto:tim.lichtenberg@rug.nl}{tim.lichtenberg@rug.nl}}} \\ 
	{\textbf{Prof. Julian Pittard} \hfill \textsl{School of Physics \& Astronomy, University of Leeds}} \\
	{\hspace*{0pt}\hfill \texttt{+44 113 343 3805}, \texttt{\href{mailto:j.m.pittard@leeds.ac.uk}{j.m.pittard@leeds.ac.uk}}}
\end{rSection}

\begin{rSection}{Publications}
	\begin{rPoints}{Papers}
		\item {Eatson, Parker \& Lichtenberg, 2024, ApJ} -- {\href{https://iopscience.iop.org/article/10.3847/1538-4357/ad8642}{\sl Towards a unified injection model of short-lived radioisotopes in $N$-body simulations of star-forming regions}}
		\item {Eatson, Lichtenberg, Parker \& Gerya, 2024, MNRAS} -- \href{https://ui.adsabs.harvard.edu/abs/2024MNRAS.528.6619E/abstract}{\sl Devolatilization of extrasolar planetesimals by \textsuperscript{60}Fe and \textsuperscript{26}Al heating} 
		\item {Eatson, Pittard \& Van Loo, 2022, MNRAS} -- {\href{https://ui.adsabs.harvard.edu/abs/2022MNRAS.517.4705E/abstract}{\sl Exploring dust growth in the episodic WCd system WR140}}
		\item {Eatson, Pittard \& Van Loo, 2022, MNRAS} -- {\sl \href{https://ui.adsabs.harvard.edu/abs/2022MNRAS.516.6132E/abstract}{An exploration of dust grain growth within WCd systems using an advected scalar dust model}}
	\end{rPoints}
	\begin{rPoints}{Conferences}
		\item {Talk: \sl Heating of planetesimals from \textsuperscript{60}Fe \& \textsuperscript{26}Al and the effect on the water content of protoplanets} -- External Irradiated Disks, Royal Society, London, UK, 2024
		\item {Poster: \sl Planetesimal heating from the SLRs \textsuperscript{60}Fe \& \textsuperscript{26}Al} -- Rocky Worlds III, Z{\"u}rich, Switzerland, 2024
		\item Poster: {\sl Devolatilization of extrasolar planetesimals by \textsuperscript{60}Fe and \textsuperscript{26}Al heating} --  Protostars \& Planets VII, Kyoto, Japan, 2023
	\end{rPoints}
	\begin{rPoints}{Other}
		\item {Doctoral thesis -- \href{https://etheses.whiterose.ac.uk/32292/}{\sl Numerical simulations of dusty Colliding Wind Binaries}} 
	\end{rPoints}
\end{rSection}

% Education section, covering university and secondary education down to GCSE level
% \begin{rSection}{Education}

% {\bf University of Leeds} \dotfill {\sl 2013-2022} \\ 
% Ph.D. in Astrophysics - {\em Numerical Simulations of Dusty Colliding Wind Binaries} \dotfill In progress \\
% MPhys in Physics \& Astrophysics \dotfill 2:1 \\
% BSc in Physics \dotfill 2:1

% {\bf Enfield Grammar School} \dotfill {\sl 2006-2013} \\
% A Level \dotfill A in History \& Mathematics, B in Physics \\
% GCSE \dotfill 13 graded C or higher, with 6 graded A

% \end{rSection}

% % Detail research projects, link to work when applicable
% \begin{rSection}{Research Projects}

% \begin{rSubsection}{Numerical Simulations of Dusty Colliding Wind Binaries}{\href{https://raw.githubusercontent.com/atomsite/Thesis/master/Thesis.pdf}{Thesis - in progress}}{Ph.D. Research Project - University of Leeds}{2017-2022}
% \item Project centred around the creation of a highly performant numerical code for performing analysis of dust formation in Colliding Wind Binary systems.
% \item Performed extensive modification to existing Athena++ and MG hydrodynamical codes to achieve this goal, co-ordination with developers of both projects, as well as general debugging and reporting.
% \item Performed parameter space exploration on requirements for dust formation in Colliding Wind Binary Systems, varying wind momentum ratio, cooling parameters and separation distance.
% \item Simulations on observed systems such as WR140 and $\eta$ Carinae performed, with particular focus on the impact of orbital eccentricity on dust formation rates.
% \item During this time developed a novel passive scalar model for simulating dust growth, destruction and cooling within a numerical simulation. Model is highly extensible and potentially applicable to a range of other hydrodynamical codes.
% \end{rSubsection}

% \begin{rSubsection}{\textit{A Comedy of Uncertainties} - Mapping Stellar Clusters Using Spatial \& Multi-Stage Sub-Clustering Methods}{\href{https://raw.githubusercontent.com/atomsite/masters-project/main/masters-report.pdf}{Project Report}}{MPhys Research Project - University of Leeds}{2016-2017}
% \item Experimentation with sub-clustering methods for application in open clusters and OB associations.
% \item Used the R statistical language to perform sub-clustering, provisional parallax data from Hipparcos-Gaia catalog was used to map stellar clusters in 3D, with the long-term goal of resolving kinematics.
% \item Initial results were promising, but conclusive results were dependent on 2\textsuperscript{nd} Gaia data release that was not available until a year after project submission.
% \end{rSubsection}

% \end{rSection}

% % Detail skills, mainly computing and research
% \begin{rSection}{Skills}

% \begin{rPoints}{Programming}
% 	\item Significant experience in several programming languages, from statistical languages such as Python, R and Julia, to low-level systems development languages such as C, C++, Fortran90 and Rust.
% 	\item Particularly fluent in C, C++, Python and R, having more than 6 years of daily usage of each language.
% 	\item Writes lean, well-documented code on-time with an emphasis on readability and parallel performance.
% 	\item Experience in modern development techniques and version control systems such as Git and Mercurial.
% 	\item Ph.D. required the understanding of HPC concepts such as shared memory and message passing parallelism, in particular the OpenMPI and OpenMP libraries.
% 	\item Familiar with other HPC concepts such as GPGPU accelerators, and have written programmes using CUDA for personal projects.
% 	\item Daily usage of IDEs such as Spyder, Rstudio, JuPyter and VSCode.
% \end{rPoints}

% \begin{rPoints}{Research Skills}
% 	\item High degree of knowledge in writing academic papers for peer review.
% 	\item Familiarity with modern documentation and static analysis methods such as Doxygen.
% 	\item Very proficient in the \LaTeX \,typesetting language as well as the Bib\TeX\, and Bib\LaTeX \,citation formats.
% 	\item Quick study for new concepts and technical jargon.
% 	\item Postgraduate level background in physics \& mathematics, IOP-accredited degree in astrophysics, with additional knowledge in numerical methods, fluid dynamics, quantum computing and computer science.
% \end{rPoints}

% \begin{rPoints}{Computing}
% 	\item Extreme proficiency with all operating systems and desktop environments, with years of experience in Windows, MacOS \& multiple Linux distributions.
% 	\item Personal experience in server maintenance and network management, as well as general technical support, typically the first point of call for most people in my department with a computing issue.
% 	\item Proficiency in office suites such as Microsoft Office, iWork and Google Workspace, as well as their open source counterparts.
% \end{rPoints}

% \begin{rPoints}{Teaching \& Collaboration}
% 	\item Experience teaching students in a wide range of age groups, from primary school to university level.
% 	\item Able to write, explain and defend concepts clearly and concisely to audiences ranging from students to seniors, educators to executives.
% 	\item Proficient in teams of any size, I work well with others, and above all else pride myself in being an asset to those that I work with.
% \end{rPoints}

% \end{rSection}

% % Skills section, but in brief, cover additional things that don't require a bullet point
% \begin{rSection}{Overview}

% {\bf Teaching} \dotfill 5 years teaching \& assessing lab skills and Python to undergraduates \\
% {\bf Fluent Programming Languages} \dotfill C, C++, Python 2.7-3.9, R \\
% {\bf Additional Programming Languages} \dotfill Fortran90, Julia, Rust \\
% {\bf Libraries \& APIs} \dotfill OpenMP, OpenMPI, Numba, Cython \\
% {\bf Practical Knowledge} \dotfill Telescope operation, server maintenance \\
% {\bf Tools \& Environments} \dotfill VSCode, JuPyter, RStudio, GNUPlot, Athena++, SGE, \LaTeX \\
% {\bf Programming Strengths} \dotfill Highly-optimised, multi-threaded code for use in HPC environments

% \end{rSection}

% % Final section on references
% \begin{rSection}{References}
% 	{\bf Dr. Julian Pittard} \dotfill {\sl Ph.D. supervisor - University of Leeds} \\
% 	\null\dotfill 0113 343 3805, {\href{mailto:J.M.Pittard@leeds.ac.uk}{J.M.Pittard@leeds.ac.uk}} \\ 
% 	{\bf Dr. Stuart Lumsden} \dotfill {\sl Masters project supervisor - University of Leeds} \\
% 	\null\dotfill 0113 343 6691, {\href{mailto:S.L.Lumsden@leeds.ac.uk}{S.L.Lumsden@leeds.ac.uk}} \\ 
% %	{\bf Mr. Charles Irvine} \dotfill {\sl Previous employer - Questions of Difference Ltd.} \\
% %	\null\dotfill {\href{mailto:charles@questionsofdifference.com}{charles@questionsofdifference.com}} \\ 
% \end{rSection}

% Finish up, hopefully this compiles!
\end{document}
